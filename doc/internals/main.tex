\documentclass{llncs}

\usepackage[utf8]{inputenc}
\usepackage[T1]{fontenc}
\usepackage[noadjust,nosort]{cite}
\usepackage{amsmath}
\usepackage{amsfonts}
%\usepackage{MnSymbol}
%\usepackage{thmtools}
%\usepackage{thm-restate}
\usepackage{booktabs}
\usepackage{figlatex}
\usepackage{xcolor}
\usepackage{setspace}
\usepackage{xspace}
\usepackage[super]{nth}
\usepackage{multicol}
\usepackage{microtype}
\usepackage{wrapfig}
\usepackage[center,width=152mm,height=235mm]{crop}
%\usepackage[pagewise,switch]{lineno} % switch, modulo, pagewise
\usepackage{hyperref}
%\usepackage[linesnumbered,lined,noend,ruled]{algorithm2e}
\usepackage[lined,noend,ruled]{algorithm2e}
\usepackage[capitalise,english,nameinlink]{cleveref} % load after algorithm2e and hyperref

\hypersetup{
  bookmarksdepth=2,
  bookmarksnumbered=true,
  bookmarksopen=true,
  bookmarksopenlevel=2,
  colorlinks=true,
  linktocpage=true,
  breaklinks=true,
  pageanchor=true,
  allcolors=[rgb]{0.6,0.0,0.0},
  pdftitle={DPU's data structures and algorithms},
  pdfauthor={Huyen N.T.T, César Rodríguez}
}

%\newcommand\comp {\mathfrak{c}}
%\newcommand\cycles[1]  {\mathsf{Cycles}{(#1)}}
%\newcommand\extn {\mathcal{X}}
%\newcommand\lchist[2]  {{ {#1}\langle#2\rangle }}
%\newcommand\lonepref   {\ppref^1_{N}}
%\newcommand\lpg     {\leadsto_\config}
%\newcommand\lstruc  {\mathcal{L}}
%\newcommand\ltwopref   {\ppref^2_{N}}
%\newcommand\maxspoilers[1]{t^{\spoils}_{#1}}
%\newcommand\move[1] {\stackrel{#1}{\longrightarrow }}
%\newcommand\myleq   \le
%\newcommand\nat     {\mathrm{I\!N}}
%\newcommand\node[1] {\mathit{#1}}
\newcommand\pac      {\mathrel{\nearrow\!\!\!\!\!\nearrow}}
\newcommand\rd       \propto
%\newcommand\spoils  {\dagger}

% relations
%\newcommand\reveals  {\mathrel{\triangleright}}
%\newcommand\ac       {\mathrel{\nearrow}}
\newcommand\aco      {\mathrel{/\!\!/}}
\newcommand\cfl      {\mathrel{\#}}
\newcommand\icfl[1][]{\mathrel{\#^i_{#1}}}
\newcommand\co       {\mathrel{\parallel}}
\newcommand\indep    {\mathrel{\meddiamond}}
\newcommand\depen    {\mathrel{\diamondtimes}}
\newcommand\eqdef    {\mathrel{:=}}
\newcommand\evolves  {\mathrel{\sqsubseteq}}
\newcommand\evstrict {\mathrel{\sqsubset}}
\newcommand\ispref   {\mathrel{\trianglelefteq}}
\newcommand\icause   {\mathrel{<_i}}
\newcommand\ifft     {\mathrel{\text{ iff }}}
\newcommand\nco      {\mathrel{\not\!\!\;\,\co}}
\newcommand\fire[1]  {\mathrel{\raisebox{-1.1pt}{$\xrightarrow{#1}$}}}

\newcommand\extreveals  {\mathbin{-\hspace{-2.5mm}-\hspace{-2.0mm}\reveals}}

% operators
\newcommand\iscutoff[1] {\mathop{\mathsf{cutoff}} (#1)}
\newcommand\union[1]    {\mathop{\mathit{union}} (#1)}
\newcommand\merge[1]    {\mathop{\mathfrak{Merge}} (#1)}
\newcommand\od[1]       {\mathop{\mathrm{od}} (#1)}
\newcommand\amo[1]      {\mathop{\textsf{AMO}} (#1)}
\newcommand\acyclic[1]  {\mathop{\textsf{ACY}} (#1)}
\newcommand\bigo[1]     {\mathop{\mathcal{O}} (#1)}
\newcommand\progloop[1] {\mathop{\mathit{ploop}} (#1)}
\newcommand\allhist[1]  {\mathop{\mathit{Hist}} (#1)}
\newcommand\compt[1]    {\mathop{\mathit{compat}} (#1)}
\newcommand\conc[1]     {\mathop{\mathit{conc}} (#1)}
\newcommand\conf[1]     {\mathop{\mathit{conf}} (#1)}
\newcommand\cut[1]      {\mathop{\mathit{cut}} (#1)}
\newcommand\explain[1]  {\mathop{\mathit{expl}} (#1)}
\newcommand\lpo[1]      {\mathop{\mathit{lpo}} (#1)}
\newcommand\marking[1]  {\mathop{\mathit{mark}} (#1)}
\newcommand\obser[1]    {\mathop{\mathit{obs}} (#1)}
\newcommand\peel[2]     {\mathop{\mathit{peel}}_{#2} (#1)}
\newcommand\peelast[1]  {\mathop{\mathit{peelmax}} (#1)}
\newcommand\reach[1]    {\mathop{\mathit{reach}} (#1)}
\newcommand\runs[1]     {\mathop{\mathit{runs}} (#1)}
\newcommand\sfp[1]      {\mathop{\mathit{pred}} (#1)}
\newcommand\spoilers[1] {\mathop{\mathit{spoilers}} (#1)}
\newcommand\state[1]    {\mathop{\mathit{state}} (#1)}
\newcommand\succexpl[1] {\mathop{\mathit{succexpl}} (#1)}
\newcommand\trim[2]     {\mathop{\mathit{trim}_{#2}} (#1)}
\newcommand\trimast[1]  {\mathop{\mathit{trimmax}} (#1)}
\newcommand\depth[1]    {\mathop{\mathit{depth}} (#1)}
\newcommand\pe[1]       {\mathop{\mathit{pe}} (#1)}
\newcommand\inter[1]    {\mathop{\mathit{inter}} (#1)}
\newcommand\proc[1]     {\mathop{\mathit{proc}} (#1)}
\newcommand\enabl[1]    {\mathop{\mathit{enabl}} (#1)}
\newcommand\en[1]       {\mathop{\mathit{en}} (#1)}
\newcommand\ex[1]       {\mathop{\mathit{ex}} (#1)}
\newcommand\cex[1]      {\mathop{\mathit{cex}} (#1)}
\newcommand\fcfl[1]     {\mathop{\mathit{\#}} (#1)}
\newcommand\ficfl[2][]  {\mathop{\mathit{\#^i_{#1} }} (#2)}
\newcommand\forml[1]    {\mathcal{L} (#1)}

\newcommand\mult[1]     {\mathbf{#1}}
\newcommand\sem[1]      {\mathrm{[\![{#1}]\!]}}
\newcommand\hist[2]     {{ {#1}[\![#2]\!] }}
\newcommand\cont[1]     {\underline{#1}}
\newcommand\post[1]     {#1^\bullet}
\newcommand\pre[1]      {{}^\bullet#1}
\newcommand\support[1]  {\bar{#1}}
\newcommand\causes[1]   {\left \lceil #1 \right \rceil}
\newcommand\future[1]   {\left \lfloor #1 \right \rfloor}

\newcommand\FIXME[1]    {\textcolor{red}{\texttt{FIXME: #1}} }
\newcommand\anc[1]      {{#1}^\uparrow}
\newcommand\eenr        {\mathcal{E}}
\newcommand\enr[1]      {\mathcal{E}_{#1}}
%\newcommand\pes         {\mathcal{E}}
\newcommand\les         {\mathcal{E}}
\newcommand\unf[1]      {\mathcal{U}_{#1}}
\newcommand\unr[1]      {\mathcal{M}_{#1}}
\newcommand\var[1]      {\mathsf{#1}}
\newcommand\mer[1]      {\mathcal{Q}_{#1}}
\newcommand\hst[1]      {\mathcal{H}_{#1}}
\newcommand\trunc[1]    {\mathcal{T}_{#1}}
\newcommand\htrunc[1]   {\widehat{\mathcal{T}}_{#1}}
\newcommand\pref[1]     {\mathcal{P}_{#1}}
\newcommand\precpref[1] {\mathcal{P}^\prec_{#1}}
\newcommand\trace[1]    {\mathcal{T}_{#1}}
\newcommand\dgraph[1]   {\mathcal{D}_{#1}}
\newcommand\intsem[1]   {\mathcal{S}_{#1}}
\newcommand\cutoffs     {\mathcal{K}}

% abbreviations
\newcommand\precutoff{\text{$\prec$-cutoff}}
\newcommand\phiasym  {\phi_\ppref^\text{asym}}
\newcommand\phicausal{\phi_\ppref^\text{causal}}
\newcommand\phiconf  {\phi_\ppref^\text{conf}}
\newcommand\phidead  {\phi_\ppref^\text{dead}}
\newcommand\phidis   {\phi_\ppref^\text{dis}}
\newcommand\phimark  {\phi_\ppref^{\text{mark},M}}
\newcommand\phicov   {\phi_\ppref^{\text{cov},M}}
\newcommand\phip     {\phi_\ppref}
\newcommand\phisym   {\phi_\ppref^\text{sym}}

\newcommand\mphimark {\gamma_\qpref^{\text{mark},M}}
\newcommand\mphicon  {\gamma_\qpref^{\text{con},M}}
\newcommand\mphiaux  {\gamma_\qpref^\text{aux}}
\newcommand\mphiflow {\gamma_\qpref^\text{flow}}
\newcommand\mphiasym {\gamma_\qpref^\text{asym}}
\newcommand\mphicov  {\gamma_\qpref^{\text{cov},M}}
\newcommand\mphigap  {\gamma_\qpref^\text{no-gap}}
\newcommand\mphipath {\gamma_\qpref^\text{path}}

\newcommand\mphiconf {\gamma_\qpref^\text{conf}}
\newcommand\mphidis  {\gamma_\qpref^\text{dis}}
\newcommand\mphip    {\gamma_\qpref}


\newcommand\mcitomp  {\textsc{Mci2mp}\@\xspace}
\newcommand\minisat  {\textsc{Minisat}\@\xspace}
\newcommand\mole     {\textsc{Mole}\@\xspace}
\newcommand\prcompress{\textsc{PRCompress}\@\xspace}
\newcommand\clp      {\textsc{Clp}\@\xspace}
\newcommand\cmerge   {\textsc{Cmerge}\@\xspace}
\newcommand\cunf     {\textsc{Cunf}\@\xspace}
\newcommand\poet     {\textsc{Poet}\@\xspace}
\newcommand\nidhugg  {\textsc{Nidhugg}\@\xspace}
\newcommand\cosyverif{\textsc{Cosyverif}\@\xspace}
\newcommand\coloane  {\textsc{Coloane}\@\xspace}
\newcommand\tapaal   {\textsc{Tapaal}\@\xspace}
\newcommand\cunft    {Cunf~Toolset}
\newcommand\cna      {\textsc{Cna}\@\xspace}
\newcommand\punf     {\textsc{Punf}\@\xspace}
\newcommand\salpha   {\ppref_\alpha}
\newcommand\spcutoff {\text{sp-cutoff}}
\newcommand\id       {\textit{id}}
\newcommand\ep       {EP}
\newcommand\dr       {Dr.\@\xspace}
\newcommand\mr       {Mr.\@\xspace}
\newcommand\prof     {Prof.\@\xspace}

\newcommand\uunf     {\mathcal{U}}
\newcommand\ppref    {\mathcal{P}}
\newcommand\qpref    {\mathcal{Q}}
\newcommand\con      {\mathcal{C}}
\newcommand\N        {\mathbb{N}}
\newcommand\Q        {\mathbb{Q}}
\newcommand\R        {\mathbb{R}}
\newcommand\obs      {\mathbb{O}}
\newcommand\hhat     {\widehat h}
\newcommand\ehat     {\widehat e}
\newcommand\chat     {\widehat c}
\newcommand\conhat   {\widehat \con}
\newcommand\Hhat     {\widehat H}
\newcommand\Ehat     {\widehat E}
\newcommand\Bhat     {\widehat B}
\newcommand\cgen     {\var{c_{\text{gen}} }}
\newcommand\cpgen    {\var{c'_{\text{gen}} }}
\newcommand\ccon     {\var{c_{\text{con}} }}
\newcommand\ccut     {\var{c_{\text{cut}} }}

\newcommand\pp       {pp.\@\xspace}
\newcommand\viz      {viz.\@\xspace}
\newcommand\vs       {vs.\@\xspace}
\newcommand\wlogg    {w.l.o.g.\xspace}
\newcommand\aka      {a.k.a.\xspace}
\newcommand\wrt      {w.r.t.\@\xspace}
\newcommand\cf       {cf.\@\xspace}
\newcommand\Wlog     {W.l.o.g.\xspace}
\newcommand\al       {al.\@\xspace}
\newcommand\eg       {e.g.\@\xspace}
\newcommand\etc      {etc.\@\xspace}
\newcommand\ie       {i.e.\@\xspace}
\newcommand\naive    {na{\"\i}ve\xspace}
%\newcommand\st      {s.t.\@\xspace}
\newcommand\resp     {resp.\@\xspace}
\newcommand\nr       {nr.\@\xspace}
\newcommand\avg      {avg.\@\xspace}

% miscellaneous
\newcommand\cod[1]   {\texttt{#1}}
\newcommand\tup[1]   {\langle#1\rangle}
\newcommand\set[1]   {{\{ #1 \mathclose \}}}
\newcommand\eqtag[1] {\hfill{} \refstepcounter{equation} \label{#1} (\arabic{equation})}

%\newtheorem{cor} {Corollary}
%\newtheorem{defn}   {Definition}
%\newtheorem{lemma}  {Lemma}
%\newtheorem{proposition}{Proposition}
%\newtheorem{remark} {Remark}
%\newtheorem{theorem}   {Theorem}

%\renewcommand{\algorithmicensure}{\textbf{Output}:}
%\renewcommand{\algorithmicrequire}{\textbf{Input}:}
%\renewcommand{\labelitemi}{\textbf{--}}
%\renewcommand{\labelitemi}{{\footnotesize $\bullet$}}

% e  equation
% d  def
% f  figure
% l  lemma, line
% s  section
% r  remark
% p  proposition
% t  theorem
% c  corollary
% b  table
% a  algorithm
% x  appendix
% h  chapter
% n  footnote
% m  example

% for standard names
\crefname{equation}{}{}
\crefname{proposition}{Prop.}{Props.}
\Crefname{proposition}{Proposition}{Propsitions}
\crefname{section}{\S}{\S\S}
\Crefname{section}{Section}{Sections}
\crefname{page}{p.}{pp.}
\Crefname{page}{Page}{Pages}
\crefname{chapter}{Ch.}{Ch.}
\Crefname{chapter}{Chapter}{Chapters}
\crefname{remark}{Rmk.}{Rmks.}
\Crefname{remark}{Remark}{Remarks}
\crefname{appendix}{App.}{App.}
\Crefname{appendix}{Appendix}{Appendix}
\crefname{algorithm}{Alg.}{Alg.}
\Crefname{algorithm}{Algorithm}{Algorithms}
%\crefname{algocf}{algorithm}{algorithm}
%\Crefname{algocf}{Algorithm}{Algorithm}
\crefname{line}{line}{line}
\Crefname{line}{Line}{Line}

% for LIPICs
\crefname{theo}{Theorem}{Theorems}
\crefname{lem}{Lemma}{Lemmas}
\crefname{cor}{Corollary}{Corollaries}
\crefname{prop}{Prop.}{Props.}
\crefname{defn}{Def.}{Defs.}
\Crefname{defn}{Definition}{Definitions}
\crefname{exampl}{Example}{Examples}
\crefname{rmk}{Remark}{Remarks}

%\crefname{AlgoLine}{algorithm}{algorithm}




\title{DPU's Data Structures and Algorithms}
%\author{Huyen Nguyen \and C\'esar Rodr\'{\i}guez}
\author{TBD}
\institute{Université Paris 13, Sorbonne Paris Cité, LIPN, CNRS, France}

\begin{document}

\maketitle

\begin{abstract}
TBD
\end{abstract}

\section{Execution Model}

Relevant classes:
\verb!Machine!,
\verb!Process!,
\verb!Trans!,
\verb!State!

Transition types :
\begin{itemize}
\item
  \verb!RD! Represents a transition that performs a certain amount of
  local (independent) work plus a single read operation on some global variable.
\item
  \verb!WR! Represents a transition that performs a certain amount of
  local (independent) work plus a single write operation on some global variable.
\item
  \verb!LOC! Represents a transition that only performs local (independent)
  operations.
\item
  \verb!SYN! Represents a synchronization operation, intended to model a
  lock or unlock operation on a mutex.
\end{itemize}

\subsection{Data Structure}

FIXME

\section{Unfolding}

Relevant classes:
\verb!Unfolding!.

\subsection{Data Structure}

A vector of \verb!Event!s, a pointer to the bottom event, and a reference
to the \verb!Machine!.

\section{Event}
Relevant classes:
\verb!Event!.

Events represent the occurrence of transitions.
They store pointers to all their immediate causal predecessors (ICPs), and some of
the immediate causal successors (ICSs).
The number of ICPs is determined by the type of the transition they
represent:

\begin{itemize}
\item \verb!LOC! events: 1 predecessor (process)
\item \verb!SYN! events: 2 predecessors (process, last \verb!SYN! on same variable)
\item \verb!RD!  events: 2 predecessor (process + last \verb!RD! or
	\verb!WR! operation on the same variable)
\item \verb!WR!  events: 1 predecessor (process) + one predecessor per
	process (last \verb!RD! or \verb!WR! operation on that process)
\end{itemize}

The immediate causal successors of one event are determined by those events
for which the event is an immediate causal predecessor. Some of the causal
successors are stored in the \verb!Event! class, but not all.

FIXME -- some of these events are only causal predecessors or successors,
not necessarily immediate predecessors or successors.

\subsection{Data Structure}

The members of the class \verb!Event! that are relevant on each instance of
the class depend on the type of the transition represented by the event.

All kinds of events store:
\begin{itemize}
\item
  \verb!trans!: a pointer to the transition it represents.
\item
  \verb!localvars!: vector (\verb!std::vector!) of \verb!uint32_t! values
  containing the value of the variables whose addresses are strored in
  \verb!this->trans->localvars!.
\item
  \verb!pre_proc!: (ICP) a pointer to the last event in the same thread.
\item
  \verb!post_proc!: (ICSs) a vector (\verb!std::vector!) of pointers to
  the next events in the same thread.
\end{itemize}
%
For \verb!SYN! events :
\begin{itemize}
\item
  \verb!pre_mem!: (ICP) pointer to the last \verb!SYN! event on the same
  variable, possibly performed by the same thread or another thread.
  Recall that \verb!SYN! operations on the same variable induce a tree when
  regarded unfolding-wise, and a total order (a branch of that tree) when
  the attention is reduced to an arbitrary configuration.
  In this regard, \verb!pre_mem! is a pointer to the node's parent in that
  tree.
\item
  \verb!post_mem!: (ICSs) vector (\verb!std::vector!) of pointers to
  the immediate next \verb!SYN! operations on the same variable, in this or
  another thread. These are the children of the aforementionned
  unfolding-wise tree.
\item
  \verb!val!: the value written by this event to the variable \verb!trans->var!
\end{itemize}
%
For \verb!RD! events :
\begin{itemize}
\item
  \verb!pre_mem!: (ICP) pointer to the last \verb!RD! or \verb!WR! event of
  the same thread. Observe that we do not store anything in the
  \verb!post_mem! vector.
\end{itemize}
%
Finally, for \verb!WR! events :
\begin{itemize}
\item
  \verb!pre_mem!: (causal predecessor) pointer to the last event
  representing a \verb!WR! operation performed on the same variable.
  Recall that \verb!WR! operations on the same variable induce a tree when
  regarded unfolding-wise, and a total order (a branch of that tree) when
  the attention is reduced to an arbitrary configuration.
  In this regard, \verb!pre_mem! is a pointer to the node's parent in that
  tree.
\item
  \verb!post_mem!: (causal successsor) vector (\verb!std::vector!) of
  pointers to the next \verb!WR! operations on the same variable, in this
  or another thread. These are the children of the aforementionned
  unfolding-wise tree.
\item
  \verb!pre_readers!: (immediate and not immediate causal predecessors)
  vector (\verb!std::vector!) of pointers to events (one per thread).
  The event pointed is the last \verb!RD! or \verb!WR! operation on the
  same variable performed in that thread.
\end{itemize}

FIXME - incomplete; look at my notes and fix


\subsection{Events in Conflict with an Event}

TBD.

\section{Configuration}
Relevant classes:
\verb!Config!.

\subsection{Data Structure}

\begin{itemize}
\item
  \verb!state!: the state reached by the configuration (type \verb!ir::State!).
\item
  \verb!latest_proc!:
  a vector of pointers to events. The size of the vector is the number of
  processes. Conceptually this represents a function from the set of processes
  to the set of events, which gives, for every process, the (unique) causally
  maximal event of that process in the configuration.

\item
  \verb!latest_wr!
  a vector of pointers to events. Size: the number of variables of the system.
  Conceptually this represents a function from the set of variables
  to the set of events, which gives, for every variable, latest (causally
  maximal) event that wrote to the variable (that event must correspond to a
  \verb!SYN! or \verb!WR! transition).

\item
  \verb!latest_op!
  a vector of pointers to events. Size: the number of variables times number of
  processes.
  Conceptually this represents a function $f \colon P \times V \to C$,
  where $P$ is the set of processes, $V$ is the set of variables, and $C$ the
  set of events in the configuration.
  The event $f (p,v)$ is the latest (causally maximal) operation that process
  $p$ performed on variable $v$ (it will correspond to a \verb!SYN!, \verb!WR!,
  or \verb!RD! transition).
\end{itemize}

\subsection{Enabled Events}

\begin{itemize}
\item
  Enumerate all transitions enabled at the \verb!state!
\item
  Using the members variables it is easy to construct new events.
\end{itemize}

\subsection{Conflicting Extensions}


Let $M$ be the input program model.
Let $\unf M \eqdef \tup{E,<,\cfl,h}$ be the unfolding of $M$.
Let $t$ be a transition of $M$ and $C \subseteq E$ a configuration of $\unf M$.

In the sequel we give an algorithm to compute the \emph{conflicting extensions}
of~$C$, defined as
$
\cex C \eqdef
\set{e \in \ex C \colon \exists e' \in C,\, e \icfl{} e'},
$
where the \emph{extensions} of~$C$, written $\ex C$,
are all those events outside~$C$ whose causes are included in~$C$.
Formally,
$
\ex C \eqdef
\set{e \in E \colon e \notin C \land \causes e \subseteq C}
$, where
$\causes e \eqdef \set{e' \in E \colon e' < e}$.

Assume we are given a configuration $C$.
We want to compute all events in $\cex C$.
Let \verb!t! be a transition of $M$.
For each type of transition we give a different algorithm that computes all
events $e \in E$ such that $h(e) = t$:
\begin{itemize}
\item Apply \cref{a:cex_rd} if $t$ is \verb!RD! transition.
\item Apply \cref{a:cex_syn} if $t$ is \verb!SYN! transition.
\item Apply \cref{a:cex_wr} if $t$ is \verb!WR! transition.
\item If $t$ is a \verb!LOC! transition, then there is no conflicting extension
$e \in \cex C$ such that $h(e) = t$.
\end{itemize}

Let $A$ be a set.
A \emph{comb} over~$A$ of size $n \in \N$ is an $n$-tuple
$c \eqdef \tup{s_1, \ldots, s_n}$ of sequences $s_i \in A^*$
over~$A$.
Each sequence $s_i$ of the $n$-tuple is called a \emph{spike}.
Given a comb $c$ of size~$n$,
a \emph{combination} of $c$ is any $n$-tuple $\tup{a_1, \ldots, a_n}$
such that $a_i$ is one of the elements of the sequence $s_i \in A^*$, for
$i \in \set{1, \ldots, n}$.

\begin{algorithm}
\noindent
For each event $e \in C$ such that \verb!e->trans == t!, repeat the following:
\begin{enumerate}
\item Set \verb!ep = e->pre_proc!
\item Set \verb!em = e->pre_mem!
\item If $\verb!em! < \verb!ep!$, then goto End.
\item If \verb!em! is \verb!RD! then set \verb!em = em->pre_mem!
\item If not it must necessarily be \verb!WR!; then set \verb!em = em->pre_readers[t->proc]!
\item (Warning: check for the bottom event in previous calculation)
\item
  Create (or retrieve) an event \verb!ex! such that
  \begin{itemize}
  \item \verb!ex->trans = t!,
  \item \verb!ex->pre_proc = ep!,
  \item \verb!ex->pre_mem = em!,
  \end{itemize}
\item Event \verb!ex! is in $\cex C$
\item Goto 3.
\item End.
\end{enumerate}
\caption{Conflicting extesions associated to \texttt{RD} transitions.}
\label{a:cex_rd}
\end{algorithm}

\begin{algorithm}
\noindent
For each event $e \in C$ such that \verb!e->trans == t!, repeat the following:
\begin{enumerate}
\item Set \verb!ep = e->pre_proc!
\item Set \verb!em = e->pre_mem!
\item If $\verb!em! < \verb!ep!$, then goto End.
\item Set \verb!em = em->pre_mem!
\item (Warning: check for the bottom event in previous calculation)
\item
  Create (or retrieve) an event \verb!ex! such that
  \begin{itemize}
  \item \verb!ex->trans = t!,
  \item \verb!ex->pre_proc = ep!,
  \item \verb!ex->pre_mem = em!,
  \end{itemize}
\item Event \verb!ex! is in $\cex C$
\item Goto 3.
\item End.
\end{enumerate}
\caption{Conflicting extesions associated to \texttt{SYN} transitions.}
\label{a:cex_syn}
\end{algorithm}

\begin{algorithm}
\noindent
UNFINISHED.
\caption{Conflicting extesions associated to \texttt{WR} transitions.}
\label{a:cex_wr}
\end{algorithm}



\section{LLVM Frontend}

\subsection{Pass 1: Sanity Checks}

Conditions to accept the input LLVM program:

\begin{itemize}
\item
  \verb!pthread_create! must be called directly from the main function and not
  in a loop.
\item
  \verb!printf!,
  \verb!pthread_{create,join,mutex_lock,mutex_unlock}! are the only function calls
  allowed (\verb!create! only in main).
\item
  All \verb!alloca! must be found in the first \verb!BasicBlock! of the
  function, and no block can jump into that block (not in a loop).
\item
  All variables in the C program are integers.
\end{itemize}

\subsection{Pass 2: Construction of 3-address IR}

\subsection{Pass 2.1: Identification of Threads}

\begin{itemize}
\item Number of threads given on command line
\item Finding out the C function executed by each thread
\end{itemize}

\subsection{Pass 2.3: Allocation of memory}

\begin{itemize}
\item
  For each global symbol (\verb!@sym!), allocated memory space (\verb!alloca!
  instruction), and local register (\verb!%sym!) we allocate space in the
  memory of our virtual machine
\item
  We generate a symbol table mapping each address to a textual name.
\end{itemize}

\subsection{Pass 2.4: Translation to 3-address IR}

Textual format and translation of LLVM opcodes in
\verb!doc/internals/3addr-ir.txt!.

\subsection{Pass 2.5: Reduction of target variables based on live variable
analysis}

Only if we have time, although this could be done based on LLVM information and
directly during the translation of LLVM instructions to ou 3-address code.

\url{https://en.wikipedia.org/wiki/Live_variable_analysis}

\subsection{Pass 3: Construction of Large-Block Encoding}

\subsection{Pass 3.1: Identification of local and global variables}
\subsection{Pass 3.2: Construction of \texttt{dpu::ir::Trans} instructions}


\section{FIXMEs}

- do we need \verb!post_*! ?


%\bibliographystyle{splncs}
%\bibliography{refs}

\end{document}

